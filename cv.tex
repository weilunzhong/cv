%% start of file `cv.tex'.
%% Based on `template.tex` from the moderncv distribution by Xavier Danaux
%
% This work may be distributed and/or modified under the
% conditions of the LaTeX Project Public License version 1.3c,
% available at http://www.latex-project.org/lppl/.


\documentclass[11pt,a4paper,sans]{moderncv}   % possible options include font size ('10pt', '11pt' and '12pt'), paper size ('a4paper', 'letterpaper', 'a5paper', 'legalpaper', 'executivepaper' and 'landscape') and font family ('sans' and 'roman')

% moderncv themes
\moderncvstyle{classic}                        % style options are 'casual' (default), 'classic', 'oldstyle' and 'banking'
\moderncvcolor{blue}                          % color options 'blue' (default), 'orange', 'green', 'red', 'purple', 'grey' and 'black'
%\renewcommand{\familydefault}{\sfdefault}    % to set the default font; use '\sfdefault' for the default sans serif font, '\rmdefault' for the default roman one, or any tex font name
\nopagenumbers{}                             % uncomment to suppress automatic page numbering for CVs longer than one page

% character encoding
%\usepackage[utf8]{inputenc}                  % if you are not using xelatex ou lualatex, replace by the encoding you are using
%\usepackage{CJKutf8}                         % if you need to use CJK to typeset your resume in Chinese, Japanese or Korean

% adjust the page margins
\usepackage[scale=0.75]{geometry}
%\setlength{\hintscolumnwidth}{3cm}           % if you want to change the width of the column with the dates
%\setlength{\makecvtitlenamewidth}{10cm}      % for the 'classic' style, if you want to force the width allocated to your name and avoid line breaks. be careful though, the length is normally calculated to avoid any overlap with your personal info; use this at your own typographical risks...

% personal data
\firstname{Weilun}
\familyname{Zhong}
%\title{A great man}                          % optional, remove / comment the line if not wanted
% \address{Nairobi}{Kenya}    % optional, remove / comment the line if not wanted
\mobile{+46 72 032 5155}                     % optional, remove / comment the line if not wanted
\email{zhongweilunmichael@gmail.com}                          % optional, remove / comment the line if not wanted
% \href{https://github.com/weilunzhong}
\homepage{www.github.com/weilunzhong}                    % optional, remove / comment the line if not wanted
%\extrainfo{additional information}            % optional, remove / comment the line if not wanted
%\quote{The quieter you become, the more you are able to hear...}                            % optional, remove / comment the line if not wanted

% to show numerical labels in the bibliography (default is to show no labels); only useful if you make citations in your resume
%\makeatletter
%\renewcommand*{\bibliographyitemlabel}{\@biblabel{\arabic{enumiv}}}
%\makeatother

% bibliography with mutiple entries
%\usepackage{multibib}
%\newcites{book,misc}{{Books},{Others}}
%----------------------------------------------------------------------------------
%            content
%----------------------------------------------------------------------------------
\begin{document}
%\begin{CJK*}{UTF8}{gbsn}                     % to typeset your resume in Chinese using CJK
%-----       resume       ---------------------------------------------------------
\makecvtitle

\section{Work Experience}
\cventry{Aug 2015 - now}{Data Engineer}{Vionlabs}{Stockholm}{Sweden}{Development of analytic platform for Streaming service\newline{}
  \begin{itemize}
  \item Processing streaming services\rq user logs to find user behavior pattern and provide better insight to user grouping and content targeting.
  \item Data processing pipeline using PySpark, Parquet and Scala.
  \item PCA(Principal Component Analysis) on content and user behavior for user segmentation.
  \item API development with RethinkDB and Tornado.
  \end{itemize}}
\cventry{Aug 2015 - Jun 2016}{Data Engineer}{Vionlabs}{Stockholm}{Sweden}{Movie feature extraction and movie recommendation system\newline{}
  \begin{itemize}
  \item Python development on feature extraction using machine learning algorithms
  \item Utilizing features for a content based recommendation system
  \item Features including image object and scene recongnition using CNN, movie end credit detection using CNN, audio emotion classification using Restricted Boltzmann Machine, autoencoder feature extraction
  \end{itemize}}
\cventry{Jan-Feb 2012}{Electrical Engineer}{FAW-Volksvagen}{Changchun}{China}{Control design of Programable Logical Controller\newline{}
  \begin{itemize}
  \item Develop robot arm motion control algorithm
  \end{itemize}}
\cventry{Feb-Aug 2011}{English Tutor}{Hangzhou}{China}{Private TOFEL instructor\newline{}
  \begin{itemize}
  \item Help sudents with reading and speaking section of the TOEFL test
  \end{itemize}}

\section{Education}
\cventry{2013-2015}{System Conrol and Robotics}{KTH}{Stockholm}{Sweden}{Master of Engineering with full scholarship, focusing on computer vision and machine learning}  % arguments 3 to 6 can be left empty
\cventry{Jul-Aug 2015}{Robotics Summer Program}{Tohoku University}{Sendai}{Japan}{Program participants focusing on interactive visual asistance}  % arguments 3 to 6 can be left empty
\cventry{2010-2013}{Electrical Engineering}{Zhejiang University}{Hangzhou}{China}{Bachelor of Electrical Engineering, focusing on automation and control}  % arguments 3 to 6 can be left empty
\cventry{Jun-Jul 2012}{Exchange Program}{Indiana University, Bloomington}{Indiana}{USA}{Exchange student focusing on cross cultural communication}  % arguments 3 to 6 can be left empty

\section{Research Activity}
\cventry{Jan-July 2015}{Master Thesis}{KTH}{Stockholm}{Sweden}{Movie scene recognition with CNN\newline{}  % arguments 3 to 6 can be left empty
  \begin{itemize}
  \item Collected movie scene dataset and trained Convolutional Neural Network with Caffe to classify movie background.
  \item Transfer the network to Tensorflow and deploy using Docker container.
  \end{itemize}}
\cventry{Sep-Dec 2014}{Autonomous Robot Project}{KTH}{Stockholm}{Sweden}{\newline{}  % arguments 3 to 6 can be left empty
  \begin{itemize}
  \item Working on the object recognition and classification part of the robot using OpenCV and PCL(Point Cloud Library) in C++.
  \item Robot mapping and localization in ROS and C++.
  \end{itemize}}
\cventry{Jul-Aug 2015}{Robotics Summer Program}{Tohoku University}{Sendai}{Japan}{\newline{}  % arguments 3 to 6 can be left empty
  \begin{itemize}
  \item Development of handheld interactive projector using OpenCV and C++.
  \end{itemize}}

\section{Programming language}
\cvitem{Large project experience}{Python, Scala, C++}
\cvitem{Limited experience}{Java, Go, Bash}

\section{Technology Summary}
\cvitem{Databases Management}{ElasticSearch, MongoDB, RethinkDB, Redius}
\cvitem{Computational framework}{Spark}
\cvitem{Machine Learning Tools}{Tensorflow, Caffe, Keras, Pylearn2, Opencv}
\cvitem{Infrastructure Tools}{Docker, Rancher, Luigi, RabbitMQ, GitLab Runner}
\cvitem{Version Control}{Git, GitLab}
\cvitem{OS}{Linux}

\section{Languages}
\cvitemwithcomment{Chinese}{Native}{Written and spoken}
\cvitemwithcomment{English}{Fluent}{Written and spoken}
\cvitemwithcomment{Swedish}{Basic}{Written and spoken}

%\renewcommand{\listitemsymbol}{-~}            % change the symbol for lists

% Publications from a BibTeX file without multibib
%  for numerical labels: \renewcommand{\bibliographyitemlabel}{\@biblabel{\arabic{enumiv}}}
%  to redefine the heading string ("Publications"): \renewcommand{\refname}{Articles}
%\nocite{*}
%\bibliographystyle{plain}
%\bibliography{publications}                   % 'publications' is the name of a BibTeX file

% Publications from a BibTeX file using the multibib package
%\section{Publications}
%\nocitebook{book1,book2}
%\bibliographystylebook{plain}
%\bibliographybook{publications}              % 'publications' is the name of a BibTeX file
%\nocitemisc{misc1,misc2,misc3}
%\bibliographystylemisc{plain}
%\bibliographymisc{publications}              % 'publications' is the name of a BibTeX file
\end{document}


%% end of file `cv.tex'.
